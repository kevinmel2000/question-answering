\documentclass[lnbip]{svmultln}
\usepackage{makeidx} % allows for indexgeneration
\usepackage{graphicx} \usepackage{amsmath} \usepackage{amssymb}
\usepackage{algorithmic} \providecommand{\norm}[1]{\lVert#1\rVert}


\begin{document}
\mainmatter % start of the contribution

\title{Question Answering System} \subtitle{NLP Project}

\titlerunning{QA System} % abbreviated title (for running head)

\author{Tudor Berariu\\Teodor Andrei T\u{a}rt\u{a}reanu\\Mihai
  Tr\u{a}sc\u{a}u\\} \authorrunning{T. Berariu, A. T\u{a}rt\u{a}reanu,
  M. Tr\u{a}sc\u{a}u} % abbreviated author list (for running head)
% \tocauthor{Tudor Berariu}
\institute{Facultat d'Inform\`{a}tica de Barcelona, \\ Universitat
  Polit\`{e}cnica de Catalunya}
\maketitle % typeset the title of the contribution

\begin{abstract}        % give a summary of your paper
  In this document we present our approach to develop a question
  answering system to work on remedia corpus. Several ideas and their
  tested utility are discussed.

  \keywords {natural language processing, question answering}
\end{abstract}

\section{Project's Goal}
\label{sec:intro}

The aim of this project was to build a question answering system
capable to answer questions referring to a small piece of text. The corpus we benchmarked our system is the Remedia Corpus (reference needed).




\bibliographystyle{plain} \bibliography{clusterbio}



\end{document}
